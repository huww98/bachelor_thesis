\chapter{跨领域上下文相关语义分析}

\section{数据集} \label{dataset}

本文主要使用最新的SParC\cite{sparc19}数据集作为主要的性能指标。它是一个大规模的,上下文相关的,具有SQL标注的语义分析数据集。

SParC数据集和之前的数据集相比有一些显著的不同。SParC数据集是基于Spider\cite{spider18}数据集构建的。Spider数据集是不包含上下文的,其中一个例子仅包含一个问题和一个SQL标注。相比于Spider数据集,SParC数据集中的每一次交互都是以Spider数据集中的一个问题作为交互目标,标注者被要求询问几个相互关联的问题来获取信息以完成该目标。其平均每次交互的轮数约为3.0轮。ATIS数据集是早些时候被广泛研究的数据集,但它所有的例子都是限制在一个领域之内的,所有交互使用同样的数据库架构。而在SParC数据集中总共包括了200个不同的数据库架构,且在测试时使用的数据库架构都是没有在训练集中出现的。这要求模型能够在推理阶段适应不同的领域。另外,SParC数据集中的问题和SQL语句也显著复杂于ATIS数据集,有些问题还需要一定的生活常识才能正确解答。

作为总结,我们之所以选择SParC数据集是因为它对模型提出了新的,更接近实际的挑战:1. 包含更复杂的自然语言上下文依赖关系;2. 覆盖了更多,更复杂的语义;3. 使用了跨领域的设置。

\section{问题定义}

定义$X$为自然语言语句,$Y$为对应的SQL查询语句,在一次用户交互$I$中,包括$n$个交互回合:$I=\{\langle X_i,Y_i\rangle\}_{i=1}^n$。定义数据库架构$T=C,T,R$,其中包含数据表列$C=\left\{c_1,c_2,c_3,\ldots,c_{\left|C\right|}\right\}$,每一个列名$c_i$包含单词$c_{i,1},\ldots,c_{i,\left|c_i\right|}$。包含数据表$T=\left\{t_1,t_2,t_3,\ldots,t_{\left|T\right|}\right\}$, 每一个表名$t_i$包含单词$t_{i,1},\ldots,t_{i,\left|t_i\right|}$。$T$中也包含主键,外键等关联信息,标记为$R$。在第$t$个交互回合,模型的目标是输入$T$,$X_t$和交互历史记录$\{\langle X_i,Y_i\rangle\}_{i=1}^{t-1}$,并生成$Y_t$。
