\chapter{结论}

\section{论文工作总结}

尽管在自然语言到SQL转换任务上已经有了不少研究,但考虑自然语言上下文的研究依然较少。本文提出的简单的上下文编码机制能与预训练模型良好结合,明显提升模型处理上下文的效率。本文提出的新的SQL预处理与后处理方法尽管较简单,但能有效降低SQL中的冗余信息,提升模型解码SQL语句的效率,最终提升任务效果。本文提出的关系编码机制能有效将预定义的关系融入到预训练的模型中。

在实践上,融入了以上改进,并受益于最新的预训练模型,本文的模型在SParC数据集上显著超过了之前最佳的结果。在构建贴近实际的数据库的自然语言接口的过程中更进了一步。

\section{工作展望}

本文对现有的SQL预处理方案做出了改进,但依然存在问题:较复杂的表关系无法在当前预处理中表达,SQL语句中依然存在一些冗余的,抽象层次过低的表达。未来我们希望能以现有ORM系统类似的方式,将数据库架构转换为对象关系模型,并将SQL表达为对该模型的操作。例如类似使用微软开发的LINQ语言对Entity Framework构建的模型进行操作。

关系编码机制虽然证实有效,但在本文的最终的实验中并未取得很显著的提升。尽管如此,我认为这种方式是非常灵活的,它可以将任意预定义的关系融入被广泛使用的注意力机制中。这个机制也能看作是注意力机制和图神经网络的结合。其能力还有待进一步挖掘,其应用领域也很可能不局限于自然语言到SQL转换任务上。

我们距离真正可用的数据库自然语言接口系统还有一定距离。比如在本文中,预测的SQL语句并不包含常量(比如自然语言查询“展示前3个人”中的“3”)。此外系统预测的准确率水平依然较低,可能难以为用户提供令人满意的服务质量。还需要更多的研究以提供改进。
