
\chapter{绪论}

\section{引言}

当今社会,各行各业都在向信息化的方向飞速发展,数据已经深入到了生活中的点点滴滴。对于专业人员来说,他们可以使用如SQL之类的计算机语言来直接地访问数据库,充分,灵活地表达自己的意图。然而对于大多数普通民众来说,他们对数据的访问依然是十分受限的,只能使用软件厂商预先设计好的用户界面来获取和分析数据。如何能让更广泛的人群更充分地利用海量的数据成为了信息化过程中的一个新的问题。

\section{研究背景}

随着信息化进程的不断加速,大数据时代的到来,数据体量的不断增大,更加突显出了人们在分析利用数据上的瓶颈。近年来自然语言理解上的研究卓有成效,个人语音助理之类的产品也已经走入了千家万户。用户普遍接受使用自然语言来表达自己的意图。开发数据库的自然语言接口正是顺应了这样的趋势。将自然语言的人机交互方式推上了更加正式的场合,将数据库的访问推向了更广阔的用户。如今数据库的自然语言接口的研究集中在将自然语言翻译为现有系统可以执行的指令,如SQL语句。这样的任务也称作自然语言到SQL转换 (Text to SQL) 任务。

新发布的标注数据集包含了自然语言问题和对应的SQL标注,这大大促进了该领域的研究,使得模型可以直接以SQL语句作为监督来训练。与之前的数据集不同,新的数据集,如WikiSQL\cite{seq2sql17}、Spider\cite{spider18}、SParC\cite{sparc19}以及中文数据集CSpider\cite{cspider19}等,在测试时使用训练时未见过的数据库架构,为模型提出了更贴近实际应用的跨领域泛化的挑战。Spider数据集中的数据库架构包含有多个表,且有主键,外键,字段类别等更丰富的结构化信息;SParC数据集则进一步引入了自然语言的上下文信息,允许用户通过多轮对话来补充,更改查询的内容。这些数据集促进了研究向更加实际的场景发展。

跨数据库的泛化能力对模型来说是个不小的挑战。早期研究中的模型是类似机器翻译任务的,直接将自然语言翻译为SQL。在跨数据库的泛化中,模型需要将数据库架构进行编码,得到所有列和表的向量表示以供解码SQL时使用。在编码的过程中,需要精心设计以充分利用数据库架构的名字、类别、主键、外键等信息。另外,模型需要识别自然语言中对数据库中列和表的引用。在不同的领域中,引用的方式也会有所不同。这个识别过程称作架构关联 (Schema linking)。

对自然语言中的上下文的处理也是关键的。实际场景中,对于较复杂的查询,用户通常需要在和系统的交互中不断调整才能完成。用户也可能调整自己的查询来探索数据的不同方面。自然语言的语义是和上下文有很强的关联的,有大量的指代和省略的现象。并且随着用户交互的进行,查询将变得越来越复杂。如何在复杂的查询中依然保持准确也是模型的一个挑战。

对输出SQL的解码工作是十分工程化的。先前的工作已经在这方面做了很多工作。如设计一个能确定性翻译到SQL的中间语言,使用SQL语法限制解码,使用SQL执行结果来指导解码等。虽然如此,但这方面依然有改进的可能:1. 相比自然语言,SQL的语义抽象程度较低,有时需要使用复杂的结构来表达相对简单的语义,之前的中间语言的设计并未完全解决此问题;2. SQL的语法在设计上更考虑的是方便人类编写和读取,对于机器则不是那么适合。

在本文中,我基于SParC数据集上的最佳方法EditSQL\cite{edit19},针对以上各方面进行改进,使得性能有了显著提高。在SParC数据集上达到了58.5\%的完全匹配准确率。

\section{研究现状}

近期新数据集的发布带动了基于语义分析方法的自然语言到SQL转换的研究。最近的研究大都以更加贴近实际的Spider和SParC数据集作为目标。在这些数据集中充分体现了上述困难和挑战。

在编码器的构造上,之前的工作主要分为两类。 \citet{gnn19}提出使用图神经网络 (GNN) 编码数据库架构信息,另外使用LSTM\cite{lstm97}编码自然语言。IRNet\cite{irnet19}使用添加了大量注意力机制的LSTM分别编码自然语言和数据库架构。RAT-SQL\cite{ratsql19}在LSTM生成的表示基础上,使用具有关系感知的自注意力机制的Transformer\cite{attn17}对自然语言和数据库架构进行协同编码。在处理数据库架构中的主键、外键关系时: \citet{gnn19}将这些关系表示为有向图,并用GNN编码它;IRNet忽略了这些关系;而RAT-SQL用自注意力机制编码这些关系。在处理架构链接时:IRNet在输入中添加特殊的“链接类型”嵌入来表达; \citet{gnn19}基于词向量嵌入和一些手工构造的特征来计算自然语言中的单词和数据库列之间的相似度; \citet{gnn19}的后续工作Global-GNN\cite{ggcn19}将自然语言查询的语义信息引入GNN的初始化中;RAT-SQL则将链接关系使用同样的自注意力机制来编码。

在解码器部分,部分研究使用模板填充的方式,将解码过程建模为多分类任务\cite{cao19,seq2sql17,sqlnet17}。
但更多近期的研究将解码过程建模为序列生成任务,它们大多使用了基于LSTM的解码器,主要有基于Token的解码\cite{edit19}和基于抽象语法树的解码\cite{irnet19,gnn19,ggcn19,syntaxsqlnet18}。
IRNet设计了一种比SQL抽象层次更高的语言。还有一些小的改进,如:由粗到细地解码\cite{irnet19,coarse-to-fine18};使用记忆增强的指针网络以减少网络生成重复的内容;使用执行结果来指导解码\cite{wang2018execution-guided},生成后进行额外的辨别步骤等\cite{ggcn19}。

\section{论文结构}

本文分为四章。其中第一章简述了数据库的自然语言接口的研究背景和意义以及自然语言到SQL转换任务的基础知识和研究现状等。第二章节从神经网络的发展历史、网络结构、学习规律三方面详细的讲述了相关神经网络的基础知识。第三章具体定义了本要解决的问题。第四章描述了本文对先前工作各方面的改进方案。第五章节是自然语言到SQL转换实验的结果与分析。
